\documentclass{article}
\usepackage{pgfplots}
\usepackage[utf8]{inputenc}
\usepackage{amsmath,amssymb,amsthm}
\usepackage[spanish]{babel}
\usepackage{graphicx,float}
\usepackage{cancel}

\pgfplotsset{compat=1.17}

\begin{document}
\section{Método de los trapecios}
El método de los trapecios es otra forma más de aproximar la integral de manera numérica, este en su forma más básica, conecta los extremos $(a,f(a))$ y $(b,f(b))$ por una línea recta y aproxima el área por el área de un trapecio,.
Este método es conceptualmente sencillo y existen tablas datadas de antes del 50 A.C. que muestran como los babilonios utilizaban este método para calcular la eclíptica de Júpiter.\cite{babilonios}%%https://www.science.org/doi/full/10.1126/science.aad8085

\begin{tikzpicture}
    \begin{axis}[
        axis lines = left,
        xlabel = $x$,
        ylabel = {$f(x)$},
        legend pos=outer north east,
        xmin=1, xmax=1.6,
    ]
    
    % Curva f(x)
    \addplot [
        domain=0:pi, 
        samples=100, 
        color=red,
    ]
    {sin(deg(x))^3};
    
    \addlegendentry{$f(x) = (\sin x)^3$}
    
    % Trapecio
    \addplot [
        fill=blue, 
        fill opacity=0.2
    ] coordinates {
        
    
        
         (1.6,0)
        (1.6,{sin(deg(1.6))^3})
        (1,{sin(deg(1))^3})
        (1,0)
        
    
    } \closedcycle;
    
    \addlegendentry{Área del trapecio}
    
    \end{axis}
    \end{tikzpicture}
    El área de un trapecio será el área del rectángulo más el área del triangulo de encima:
    
\begin{equation}\label{eq2}
A_{\text {trapecio}} = \underbrace{(b-a)f(a)}_{\text{área triangulo}} + \underbrace{\frac{(a-b)(f(b)-f(a))}{2}}_{\text{área trapecio}} = (a-b)\frac{f(a)+f(b)}{2}
\end{equation}
\begin{equation}\label{eq2}
A_{\text {función}} \thickapprox (b-a)\frac{f(a) + f(b)}{2}
\end{equation}


Esto según se ve en la figura 2a puede parecer algo poco preciso, es por esto que aumentando el número de trapecios obtenemos una mayor precisión usando así la regla de los trapecios compuesta.



La regla de los Trapecios compuesta se usa de la siguiente manera:
    \begin{align*}
        &\text{Dividimos } [a,b] \text{ en } n \text{ subintervalos definidos por } n + 1 \text{ puntos} \\
        &a =x_0 < x_1 < \cdots < x_{n-1} < x_n = b , \quad \Delta x_j =x_j - x_{j-1} \\
        &\int_{a}^{b} f (x)\,dx \approx \sum_{j=1}^{n} \text{Area}(T_j) =\sum_{j=1}^{n}
          \frac{f (x_{j-1}) + f (x_j)}{2} \Delta x_j \\
        &\text{Si la anchura de los trapecios es igual, } \Delta x_j = \Delta x = \frac{b - a}{n} ,\text{entonces:} \\
        &\int_{a}^{b} f (x)\,dx \approx \frac{\Delta x}{2}\left[f(a)+2\sum_{j=1}^{n-1}f(x_j)+f(b)  \right] \\
        &\Delta x \left[ \frac{f(a)+f(b)}{2} + \sum_{j=1}^{n-1} f (x_j)  \right] = T_n
        \end{align*}


            %FIGURA 2
            \begin{tikzpicture}
                \begin{axis}[
                    axis lines = left,
                    xlabel = $x$,
                    ylabel = {$f(x)$},
                    legend pos=outer north east,
                    xmin=1, xmax=1.6,
                ]
                
                % Curva f(x)
                \addplot [
                    domain=0:pi, 
                    samples=100, 
                    color=red,
                ]
                {sin(deg(x))^3};
                
                \addlegendentry{$f(x) = (\sin x)^3$}
                
                % Trapecio
                \addplot [
                    fill=blue, 
                    fill opacity=0.2
                ] coordinates {
                    
                
                    
                     (1.2,0)
                    (1.2,{sin(deg(1.2))^3})
                    (1,{sin(deg(1))^3})
                    (1,0)
                    
                
                } \closedcycle;
                \addplot [
                    fill=blue, 
                    fill opacity=0.2
                ] coordinates {
                    
                
                    
                     (1.4,0)
                    (1.4,{sin(deg(1.4))^3})
                    (1.2,{sin(deg(1.2))^3})
                    (1.2,0)
                    
                
                } \closedcycle;
                \addplot [
                    fill=blue, 
                    fill opacity=0.2
                ] coordinates {
                    
                
                    
                     (1.6,0)
                    (1.6,{sin(deg(1.6))^3})
                    (1.4,{sin(deg(1.4))^3})
                    (1.4,0)
                    
                
                } \closedcycle;
                
                
                \addlegendentry{Área del trapecio}
                
                \end{axis}
                \end{tikzpicture}
                Como podemos observar de esta manera nuestra aproximación del área es mucho más precisa solamente dividiendo en 3 intervalos.
        De hecho, el error absoluto cometido en la aproximación por los trapecios está acotado por:
$$
E_n = \left| \int_{a}^{b} f(x)dx  - T_n\right| \leq \frac{K_2(b-a)^3}{12n^2}= \frac{K_2(b-a)}{12}(\Delta x)^2 \sim O(\Delta x^2)
$$






donde:
$$
K_2 = \max_{x \in [a, b]} |f''(x)|
$$
\begin{proof}
Si dividimos $[a,b]$ en $n$ subintervalos definidos por los $n + 1$ puntos:
$a =x_0 <x_1 <\cdots < x_{n-1} < x_n = b$ , $\Delta x = \frac{b-a}{n}$, tenemos:
\[
\int_{a}^{b} f (x)dx = \sum_{k=1}^{n} \int_{x_{k-1}}^{x_k} f (x)dx
\]
En cada intervalo $[x_{k-1},x_k]$ se aplica la regla del trapecio simple:
\[
\int_{a}^{b} f (x)dx = \sum_{k=1}^{n} \left[\frac{f (x_{k-1}) + f (x_k)}{2}\Delta x - \frac{f''(\xi_k)}{12} (\Delta x)^3\right]
\]
\[
= T_n - \frac{f''(\xi)}{12} \sum_{k=1}^{n} (\Delta x)^3 = T_n - \frac{f''(\xi)}{12} n(\Delta x)^3
\]
Como $\Delta x = \frac{b-a}{n}$, entonces:
\[
\int_{a}^{b} f (x)dx = T_n - \frac{f''(\xi)(b -a)^3}{12n^2}
\]
Finalmente, el error absoluto cometido está acotado por:
\[
E = \left|\int_{a}^{b} f (x)dx -T_n\right| = \left|\frac{f''(\xi)(b -a)^3}{12n^2}\right| = \frac{(b-a)^3}{12n^2} |f''(\xi)| \leq \frac{K_2(b -a)^3}{12n^2} = \frac{K_2(b-a)}{12} (\Delta x)^2
\]
donde:
\[
K_2 = \max_{x\in[a,b]} |f''(x)| 
\]
\end{proof}


        
        
    \end{document}
    




